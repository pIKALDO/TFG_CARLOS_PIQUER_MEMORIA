% ======================================
% CAPÍTULO 2 - ESTADO DEL ARTE
% ======================================

\section{Análisis de aplicaciones similares}
En el mercado actual la gran variedad de plataformas que nos ofrecen servicios para
crear nuestra propia tienda online sin necesidad de tener algún tipo de conocimiento sobre
programar es muy extenso. Algunas de las más representativas son:

\subsection{Wix eCommerce}
Wix proporciona un sistema de creación de páginas basado en el uso de herramientas
de arrastrar y soltar, lo que facilita a cualquier usuario construir una tienda en poco
tiempo \cite{wixHelp}. Sin embargo, esta simplicidad también conlleva limitaciones, especialmente en lo
que respecta a la personalización avanzada y al control sobre el backend, aspectos que resultan esenciales en proyectos con requisitos específicos.

\textbf{Ventajas:}
\begin{itemize}
    \item Interfaz intuitiva y sistema de diseño drag-and-drop.
    \item Plantillas atractivas y actualizadas.
    \item Incluye hosting y mantenimiento automático.
    \item Integración sencilla con herramientas externas.
\end{itemize}

\textbf{Desventajas:}
\begin{itemize}
    \item Escasa flexibilidad para personalizaciones complejas.
    \item Limitaciones en el acceso al código fuente.
    \item Dificultades para escalar proyectos grandes.
    \item Dependencia total del ecosistema de Wix.
\end{itemize}

\subsection{Squarespace}
Squarespace está orientado a usuarios que priorizan el diseño visual y la facilidad de
uso \cite{squarespaceHelp}. Sus plantillas resultan intuitivas y permiten crear páginas atractivas sin dificultad. No
obstante, su ecosistema de integraciones es más limitado, lo que restringe su aplicabilidad
en proyectos que requieren funcionalidades técnicas más avanzadas.

\textbf{Ventajas:}
\begin{itemize}
    \item Excelente diseño visual y acabados profesionales.
    \item Plataforma todo en uno (hosting, dominio y soporte).
    \item Adaptación automática a dispositivos móviles.
    \item Seguridad gestionada por la propia plataforma.
\end{itemize}

\textbf{Desventajas:}
\begin{itemize}
    \item Pocas integraciones con APIs o sistemas externos.
    \item Costes mensuales relativamente altos.
    \item Escasa libertad para modificar la estructura del sitio.
    \item Dependencia del entorno cerrado de Squarespace.
\end{itemize}

\subsection{PrestaShop}
PrestaShop es una solución de código abierto muy popular para la creación de tiendas
online \cite{prestashopDocs}. Ofrece una gran capacidad de personalización mediante módulos y plantillas, lo que la convierte en una opción atractiva para pequeñas y medianas empresas. No obstante, requiere ciertos conocimientos técnicos para su instalación y mantenimiento, y en proyectos de mayor tamaño puede presentar dificultades de rendimiento si no se configura correctamente.

\textbf{Ventajas:}
\begin{itemize}
    \item Open source y altamente personalizable.
    \item Gran comunidad de desarrolladores.
    \item Amplio catálogo de módulos y temas.
    \item Permite control total sobre el servidor y la base de datos.
\end{itemize}

\textbf{Desventajas:}
\begin{itemize}
    \item Necesita conocimientos técnicos para instalación y soporte.
    \item Requiere mantenimiento y actualizaciones manuales.
    \item Puede presentar problemas de rendimiento en tiendas grandes.
    \item Algunos módulos son de pago y elevan los costes.
\end{itemize}

\subsection{Comparación con la propuesta del TFG}
Las plataformas analizadas muestran distintas formas de crear una tienda online sin necesidad de conocimientos de programación. Wix y Squarespace destacan por su sencillez
y diseño intuitivo, aunque sacrifican flexibilidad y control sobre la aplicación. Por su parte, PrestaShop ofrece mayor capacidad de personalización, pero exige un nivel técnico más elevado y puede presentar limitaciones de rendimiento en proyectos de gran escala.
En contraste, la propuesta de este TFG plantea el desarrollo de una solución a medida  que permita un mayor control sobre la arquitectura, la seguridad y las integraciones, 
adaptándose específicamente a las necesidades del proyecto.


\begin{table}[H]
\centering
\begin{tabular}{|l|c|c|c|}
\hline
\textbf{Criterio} & \textbf{Wix} & \textbf{Squarespace} & \textbf{PrestaShop} \\ \hline
Facilidad de uso & Muy alta & Alta & Media \\ \hline
Personalización & Baja & Media & Alta \\ \hline
Escalabilidad & Limitada & Media & Alta \\ \hline
Requiere conocimientos técnicos & No & No & Sí \\ \hline
Coste & Medio & Medio & Variable \\ \hline
\end{tabular}
\caption{Comparativa de plataformas de creación de tiendas online}
\end{table}


\section{Tecnologías}

\subsection{Frameworks frontend}

\subsubsection{Angular}
Angular es un framework desarrollado por Google y basado en TypeScript \cite{angularDocs}. Ofrece una solución completa para el desarrollo de aplicaciones web a gran escala, con herramientas integradas como enrutamiento y gestión de formularios. Su principal desventaja es la necesidad de contar con conocimientos previos más avanzados, lo que incrementa la curva de aprendizaje.

\textbf{Ventajas:}
\begin{itemize}
    \item Framework completo con soporte oficial de Google.
    \item Tipado fuerte con TypeScript, que mejora la robustez del código.
    \item Soporte integrado para testing, formularios y enrutamiento.
    \item Comunidad amplia y documentación exhaustiva.
\end{itemize}

\textbf{Desventajas:}
\begin{itemize}
    \item Curva de aprendizaje pronunciada.
    \item Tamaño inicial del proyecto elevado.
    \item Requiere experiencia previa en TypeScript.
    \item Sintaxis más compleja frente a otras alternativas.
\end{itemize}

\subsubsection{React}
React es una tecnología desarrollada por Meta \cite{reactDocs}. Se trata de una biblioteca para la construcción de interfaces de usuario, que destaca por su flexibilidad y por el rendimiento que ofrece gracias al uso del Virtual DOM. Sin embargo, al no ser un framework completo, depende de librerías de la comunidad para cubrir funcionalidades adicionales o más avanzadas.

\textbf{Ventajas:}
\begin{itemize}
    \item Excelente rendimiento gracias al Virtual DOM.
    \item Gran comunidad y recursos de aprendizaje.
    \item Compatible con múltiples librerías y frameworks.
    \item Curva de aprendizaje accesible para principiantes.
\end{itemize}

\textbf{Desventajas:}
\begin{itemize}
    \item Necesidad de integrar herramientas externas (routing, estado).
    \item Menor estructura definida que Angular.
    \item Frecuentes actualizaciones y cambios en librerías.
    \item Requiere experiencia para mantener proyectos grandes.
\end{itemize}

\subsubsection{Vue}
Vue es un framework progresivo creado por Evan You, exingeniero de Google \cite{vueDocs}. Combina características de Angular y React, con un enfoque en la simplicidad y en la rapidez de aprendizaje. No obstante, su comunidad y ecosistema siguen siendo más reducidos en comparación con otras tecnologías consolidadas.

\textbf{Ventajas:}
\begin{itemize}
    \item Ligero y de fácil adopción.
    \item Documentación clara y organizada.
    \item Buena integración con proyectos existentes.
    \item Sintaxis sencilla y curva de aprendizaje rápida.
\end{itemize}

\textbf{Desventajas:}
\begin{itemize}
    \item Ecosistema más limitado que Angular o React.
    \item Menor presencia en grandes empresas.
    \item Menos recursos de formación avanzada.
    \item Riesgo de menor soporte a largo plazo.
\end{itemize}

\subsubsection{Comparación y elección}
Los tres frameworks permiten crear aplicaciones modernas, pero Angular ofrece una
estructura más completa desde el inicio. Para este TFG se ha optado por Angular, ya que
es la tecnología utilizada en Capgemini, empresa en la que voy a realizar las prácticas. De
esta forma, el proyecto servirá también como una preparación práctica, complementada
con cursos de formación, para no comenzar esa experiencia sin conocimientos previos.

\begin{table}[H]
\centering
\begin{tabular}{|l|c|c|c|}
\hline
\textbf{Criterio} & \textbf{Angular} & \textbf{React} & \textbf{Vue} \\ \hline
Lenguaje principal & TypeScript & JavaScript & JavaScript \\ \hline
Complejidad inicial & Alta & Media & Baja \\ \hline
Ecosistema & Muy amplio & Amplio & Moderado \\ \hline
Soporte empresarial & Alto & Alto & Medio \\ \hline
Rendimiento general & Alto & Muy alto & Alto \\ \hline
\end{tabular}
\caption{Comparativa de frameworks frontend}
\end{table}


\subsection{Frameworks backend}

\subsubsection{Spring Boot}
Spring Boot es un framework del ecosistema Java que facilita la creación de aplicaciones backend robustas y escalables \cite{springDocs}. Entre sus principales ventajas se encuentran su amplia comunidad, la integración con múltiples librerías y la facilidad para configurar proyectos complejos. Como inconvenientes, requiere conocimientos previos en Java y, según el tamaño de la aplicación, puede resultar más pesado que otras alternativas.

\textbf{Ventajas:}
\begin{itemize}
    \item Framework maduro con gran soporte empresarial.
    \item Integración nativa con bases de datos y APIs REST.
    \item Facilita pruebas y despliegues automáticos.
    \item Gran estabilidad y seguridad.
\end{itemize}

\textbf{Desventajas:}
\begin{itemize}
    \item Requiere conocimientos avanzados en Java.
    \item Consumo de recursos más alto que Node.js.
    \item Curva de aprendizaje larga.
    \item Configuraciones iniciales complejas.
\end{itemize}

\subsubsection{Node.js}
Node.js es un entorno de ejecución de JavaScript que permite utilizar este lenguaje en la configuración y desarrollo de servidores \cite{nodeDocs}. Destaca por su rendimiento en aplicaciones en tiempo real y por la ventaja de compartir el mismo lenguaje en frontend y backend, lo que agiliza el desarrollo. Su principal limitación es que no está tan orientado a aplicaciones de gran escala como otros frameworks más consolidados.

\textbf{Ventajas:}
\begin{itemize}
    \item Excelente rendimiento en tiempo real.
    \item Usa el mismo lenguaje en cliente y servidor.
    \item Amplia comunidad y soporte.
    \item Ecosistema extenso con NPM.
\end{itemize}

\textbf{Desventajas:}
\begin{itemize}
    \item Menor rendimiento en tareas muy pesadas.
    \item Dificultad para proyectos altamente estructurados.
    \item Depende mucho de librerías externas.
    \item Escalabilidad más limitada que Spring Boot.
\end{itemize}

\subsubsection{Django}
Django es un framework de Python que sigue el principio "batteries included", lo que significa que incorpora numerosas funcionalidades listas para usar \cite{djangoDocs}. Es adecuado para el desarrollo rápido de prototipos y proyectos de tamaño medio, con una curva de aprendizaje moderada. Sin embargo, su rendimiento puede verse limitado en aplicaciones muy exigentes.

\textbf{Ventajas:}
\begin{itemize}
    \item Gran cantidad de herramientas integradas.
    \item Excelente para desarrollo rápido.
    \item Seguridad y control de autenticación robusto.
    \item Documentación muy completa.
\end{itemize}

\textbf{Desventajas:}
\begin{itemize}
    \item Menor rendimiento en aplicaciones grandes.
    \item Menor integración con sistemas Java.
    \item No tan extendido en entornos empresariales.
    \item Difícil de optimizar para proyectos complejos.
\end{itemize}

\subsubsection{Comparación y elección}
Spring Boot, Node.js y Django son alternativas viables para el desarrollo de un backend. Node.js destaca por su rapidez en tiempo real y Django por la simplicidad de Python, pero para este TFG se ha optado por Spring Boot. Esta elección responde a que es la tecnología utilizada en Capgemini, empresa en la que realizaré las prácticas, lo
que me permite formarme previamente mediante cursos y adquirir los conceptos básicos necesarios. Además, Spring Boot destaca por su robustez, su madurez en entornos empresariales y porque complementa de forma óptima la arquitectura planteada con Angular y SQL.

\begin{table}[H]
\centering
\begin{tabular}{|l|c|c|c|}
\hline
\textbf{Criterio} & \textbf{Spring Boot} & \textbf{Node.js} & \textbf{Django} \\ \hline
Lenguaje base & Java & JavaScript & Python \\ \hline
Escalabilidad & Alta & Media & Media \\ \hline
Curva de aprendizaje & Alta & Media & Baja \\ \hline
Rendimiento general & Alto & Alto & Medio \\ \hline
Soporte empresarial & Muy alto & Medio & Medio \\ \hline
\end{tabular}
\caption{Comparativa de frameworks backend}
\end{table}


\subsection{Agentes de automatización}

\subsubsection{Zapier}
Zapier es una herramienta para automatizar tareas mediante la conexión de aplicaciones. Destaca por su simplicidad y facilidad de uso, aunque sus funcionalidades avanzadas requieren una suscripción de pago\cite{zapierDocs}.

\textbf{Ventajas:}
\begin{itemize}
    \item Muy fácil de usar.
    \item Gran catálogo de aplicaciones compatibles.
    \item Ideal para tareas simples y rápidas.
    \item Interfaz visual clara y moderna.
\end{itemize}

\textbf{Desventajas:}
\begin{itemize}
    \item Las funciones avanzadas requieren suscripción.
    \item Limitado control sobre la configuración interna.
    \item Escasa flexibilidad para integraciones complejas.
    \item No permite autoalojamiento.
\end{itemize}

\subsubsection{Make}
Make, anteriormente conocido como Integromat, es una herramienta que ofrece un
sistema visual para crear flujos de trabajo más complejos que en Zapier. Es potente y
flexible, aunque su uso intensivo también depende de planes de pago y de la integración
con servicios externos\cite{makeDocs}.

\textbf{Ventajas:}
\begin{itemize}
    \item Muy flexible para automatizaciones avanzadas.
    \item Interfaz visual potente y configurable.
    \item Permite múltiples pasos por flujo.
    \item Buen equilibrio entre facilidad y personalización.
\end{itemize}

\textbf{Desventajas:}
\begin{itemize}
    \item Algunas funciones requieren plan de pago.
    \item Requiere conexión constante a la nube.
    \item Mayor complejidad de uso frente a Zapier.
    \item Menor documentación en español.
\end{itemize}

\subsubsection{n8n}
n8n es una plataforma de automatización de código abierto que permite crear flujos personalizados e instalarlos en servidores propios, sin necesidad de planes de pago. Ofrece un gran control sobre la configuración y resulta una herramienta intuitiva y fácil de utilizar \cite{n8nDocs}.

\textbf{Ventajas:}
\begin{itemize}
    \item Totalmente open source y gratuita.
    \item Puede instalarse en servidores propios.
    \item Gran control sobre la configuración.
    \item Compatible con múltiples servicios y APIs.
\end{itemize}

\textbf{Desventajas:}
\begin{itemize}
    \item Requiere configuración inicial manual.
    \item Menor comunidad que Zapier.
    \item No tan intuitiva para principiantes.
    \item Necesita recursos del servidor donde se aloje.
\end{itemize}

\subsubsection{Comparación y elección}
Zapier y Make destacan por su facilidad de uso y popularidad, pero dependen de planes de suscripción y de servicios externos, además de no ofrecer un control total sobre la configuración. En este TFG se ha optado por n8n por ser una solución open source, gratuita y altamente configurable, lo que facilita su integración con la arquitectura  propuesta.

\begin{table}[H]
\centering
\begin{tabular}{|l|c|c|c|}
\hline
\textbf{Criterio} & \textbf{Zapier} & \textbf{Make} & \textbf{n8n} \\ \hline
Tipo de licencia & De pago & Freemium & Open source \\ \hline
Facilidad de uso & Muy alta & Alta & Media \\ \hline
Control y personalización & Bajo & Medio & Alto \\ \hline
Coste & Elevado & Medio & Gratuito \\ \hline
Soporte técnico & Alto & Medio & Medio \\ \hline
\end{tabular}
\caption{Comparativa de agentes de automatización}
\end{table}


\subsection{Bases de datos}

\subsubsection{SQL}
Las bases de datos relacionales, como MySQL o PostgreSQL, utilizan un modelo basado en tablas y relaciones entre ellas \cite{mysqlDocs}. Destacan por su consistencia, su alto nivel de seguridad y por el uso del lenguaje SQL, que es un estándar ampliamente adoptado en el entorno empresarial.

\subsubsection{NoSQL}
Las bases de datos NoSQL, como MongoDB, almacenan la información en estructuras más flexibles, como documentos o grafos. Ofrecen mayor escalabilidad y rapidez en ciertas operaciones, aunque sacrifican consistencia y mecanismos de seguridad más avanzados. \cite{mongodbDocs}.

\subsubsection{Comparación y elección}
Para este TFG se ha optado por utilizar \textbf{MySQL} como sistema de gestión de bases de datos. La elección se debe a su estabilidad, su integración nativa con Spring Boot y su capacidad para garantizar la seguridad de los datos. Además, a lo largo de la carrera se ha trabajado ampliamente con bases de datos SQL, mientras que NoSQL apenas se ha tratado, por lo que MySQL resulta una opción más familiar y adecuada para el desarrollo del proyecto.

\begin{table}[H]
\centering
\begin{tabular}{|l|c|c|}
\hline
\textbf{Criterio} & \textbf{SQL (MySQL)} & \textbf{NoSQL (MongoDB)} \\ \hline
Estructura & Tablas relacionales & Documentos / grafos \\ \hline
Consistencia & Alta & Media \\ \hline
Seguridad & Alta & Media \\ \hline
Escalabilidad & Media & Alta \\ \hline
Soporte empresarial & Muy alto & Medio \\ \hline
\end{tabular}
\caption{Comparativa entre bases de datos SQL y NoSQL}
\end{table}

\subsection{Pasarelas de pago}

\subsubsection{PayPal}
PayPal es una de las plataformas de pago online más conocidas y utilizadas a nivel mundial \cite{paypalDocs}. Permite realizar transacciones de forma sencilla y segura, tanto con tarjeta como con saldo en cuenta. Es muy popular entre los usuarios por su rapidez y confianza, aunque cobra comisiones algo más altas que otras alternativas.

\textbf{Ventajas:}
\begin{itemize}
    \item Gran confianza y reconocimiento internacional.
    \item Permite pagar sin necesidad de introducir datos bancarios en cada compra.
    \item Interfaz sencilla para el usuario.
    \item Compatible con muchas plataformas de comercio electrónico.
\end{itemize}

\textbf{Desventajas:}
\begin{itemize}
    \item Comisiones elevadas por transacción.
    \item Poca flexibilidad para personalizar el flujo de pago.
    \item Integración técnica menos avanzada que otras opciones.
    \item Posibles retenciones temporales de fondos en algunas operaciones.
\end{itemize}


\subsubsection{Redsys}
Redsys es la pasarela de pago más extendida en España y se utiliza en la mayoría de tiendas online que trabajan con bancos nacionales \cite{redsysDocs}. Ofrece un alto nivel de seguridad y cumple con las normativas europeas, aunque su configuración inicial puede ser más compleja y su documentación menos accesible para desarrolladores.

\textbf{Ventajas:}
\begin{itemize}
    \item Muy utilizada en España, con soporte por parte de los principales bancos.
    \item Alta seguridad gracias al uso de sistemas 3D Secure.
    \item Cumple con las normativas PSD2 y SCA.
    \item Fiable y con buena reputación en entornos comerciales.
\end{itemize}

\textbf{Desventajas:}
\begin{itemize}
    \item Configuración inicial compleja.
    \item Documentación técnica limitada.
    \item Poco orientada a desarrolladores.
    \item Limitada flexibilidad en integraciones personalizadas.
\end{itemize}


\subsubsection{Stripe}
Stripe es una pasarela de pago moderna pensada especialmente para desarrolladores \cite{stripeDocs}. Su API es clara y bien documentada, lo que facilita mucho la integración con aplicaciones web. Permite gestionar pagos con tarjeta, transferencias y otros métodos de forma segura y rápida. Es una opción muy utilizada por startups y empresas tecnológicas por su facilidad de uso y sus amplias posibilidades de personalización.

\textbf{Ventajas:}
\begin{itemize}
    \item API muy completa y fácil de integrar.
    \item Admite múltiples métodos de pago y divisas.
    \item Buen equilibrio entre seguridad y personalización.
    \item Excelente documentación y comunidad activa.
\end{itemize}

\textbf{Desventajas:}
\begin{itemize}
    \item Requiere conocimientos técnicos básicos para la configuración.
    \item Comisiones similares a las de PayPal.
    \item Depende de una conexión estable con la API.
    \item No disponible en todos los países.
\end{itemize}


\subsubsection{Comparación y elección}
PayPal, Redsys y Stripe son tres opciones válidas para integrar pagos en una tienda online. PayPal destaca por su popularidad y facilidad de uso, mientras que Redsys es la alternativa más común en España y ofrece gran seguridad. Sin embargo, para este TFG se ha optado por \textbf{Stripe} debido a su flexibilidad, buena documentación y facilidad de integración con tecnologías modernas como Spring Boot y Angular. Además, su API REST simplifica la conexión con el backend y garantiza un proceso de pago seguro y fluido para el usuario.

\begin{table}[H]
\centering
\begin{tabular}{|l|c|c|c|}
\hline
\textbf{Criterio} & \textbf{PayPal} & \textbf{Redsys} & \textbf{Stripe} \\ \hline
Popularidad & Muy alta & Alta & Alta \\ \hline
Facilidad de integración & Media & Baja & Alta \\ \hline
Comisiones & Altas & Medias & Medias \\ \hline
Orientación a desarrolladores & Baja & Baja & Muy alta \\ \hline
Seguridad & Alta & Muy alta & Alta \\ \hline
\end{tabular}
\caption{Comparativa de pasarelas de pago}
\end{table}


