\documentclass[twoside,spanish,a4paper,12pt]{tfg}

% Editar la titulación
\titulacion{Grado en Ingeniería \\ Informática}

% Editar el título
\title{Desarrollo de una tienda online para una droguería}

% Si es una alumna se debe usar
% \authorlabel{Autora}
\authorlabel{Autor}
% Editar el nombre
\author{Carlos Javier Piquer Mañez}


% Si hay varios tutores:
% \tutorlabel{Tutores}
% \tutor{Nombre del tutor 1 \\[2mm] Nombre del turor2}
% Si el tutor es masculino:
% \tutorlabel{Tutor}
\tutorlabel{Tutor}
% Editar
\tutor{Andres Rocafull Rodrigo} 

% Editar: Poner mes y año de la convocatoria de lectura del TFM
\convocatoria{Marzo 2026}

\usepackage{float}
\usepackage{array}
\newcolumntype{L}[1]{>{\raggedright\arraybackslash}p{#1}}
\usepackage{graphicx}  
\usepackage{rotating} 
\usepackage{pdflscape}


\begin{document}

% NO QUITAR ESTOS ELEMENTOS
\portada
\cleardoublepage
\contraportada
\cleardoublepage
\declaracion
\cleardoublepage


% Editar: Resumen en Español (obligatorio)
\begin{resumen}
  Este es el resumen del TFM. Debe ser corto (máximo media página) y cubrir los aspectos principales del TFM.
\end{resumen}
\cleardoublepage

% Editar: Resumen en Inglés
\begin{abstract}
  This is the abstract of the TFM. It must be short and cover the main aspects of the TFM.
\end{abstract}
\cleardoublepage

% Editar: Resumen en Valenciano
\begin{resum}
  Aquest és el resum del TFM. Ha de ser curt (màxim mitja pàgina) i cobrir els aspectes principals del TFM.
\end{resum}
\cleardoublepage


% Editar: Agradecimientos (opcional)
\begin{agradecimientos}
  En primer lugar quiero agradecer a todos aquellos que me han apoyado durante todos estos años.

  En segundo lugar...
\end{agradecimientos}
\cleardoublepage

\tableofcontents

\pagestyle{tfg}
\justify

% Las figuras se buscan en el directorio figs

% Cada capítulo está en su propio fichero tex. Ver el directorio tex.

% La bibliografía está dentro del directorio bib
\chapter{Introducción}
\input{tex/introduccion.tex}

\chapter{Estado del arte}
% ======================================
% CAPÍTULO 2 - ESTADO DEL ARTE
% ======================================

\section{Análisis de aplicaciones similares}
En el mercado actual la gran variedad de plataformas que nos ofrecen servicios para
crear nuestra propia tienda online sin necesidad de tener algún tipo de conocimiento sobre
programar es muy extenso. Algunas de las más representativas son:

\subsection{Wix eCommerce}
Wix proporciona un sistema de creación de páginas basado en el uso de herramientas
de arrastrar y soltar, lo que facilita a cualquier usuario construir una tienda en poco
tiempo \cite{wixHelp}. Sin embargo, esta simplicidad también conlleva limitaciones, especialmente en lo
que respecta a la personalización avanzada y al control sobre el backend, aspectos que resultan esenciales en proyectos con requisitos específicos.

\textbf{Ventajas:}
\begin{itemize}
    \item Interfaz intuitiva y sistema de diseño drag-and-drop.
    \item Plantillas atractivas y actualizadas.
    \item Incluye hosting y mantenimiento automático.
    \item Integración sencilla con herramientas externas.
\end{itemize}

\textbf{Desventajas:}
\begin{itemize}
    \item Escasa flexibilidad para personalizaciones complejas.
    \item Limitaciones en el acceso al código fuente.
    \item Dificultades para escalar proyectos grandes.
    \item Dependencia total del ecosistema de Wix.
\end{itemize}

\subsection{Squarespace}
Squarespace está orientado a usuarios que priorizan el diseño visual y la facilidad de
uso \cite{squarespaceHelp}. Sus plantillas resultan intuitivas y permiten crear páginas atractivas sin dificultad. No
obstante, su ecosistema de integraciones es más limitado, lo que restringe su aplicabilidad
en proyectos que requieren funcionalidades técnicas más avanzadas.

\textbf{Ventajas:}
\begin{itemize}
    \item Excelente diseño visual y acabados profesionales.
    \item Plataforma todo en uno (hosting, dominio y soporte).
    \item Adaptación automática a dispositivos móviles.
    \item Seguridad gestionada por la propia plataforma.
\end{itemize}

\textbf{Desventajas:}
\begin{itemize}
    \item Pocas integraciones con APIs o sistemas externos.
    \item Costes mensuales relativamente altos.
    \item Escasa libertad para modificar la estructura del sitio.
    \item Dependencia del entorno cerrado de Squarespace.
\end{itemize}

\subsection{PrestaShop}
PrestaShop es una solución de código abierto muy popular para la creación de tiendas
online \cite{prestashopDocs}. Ofrece una gran capacidad de personalización mediante módulos y plantillas, lo que la convierte en una opción atractiva para pequeñas y medianas empresas. No obstante, requiere ciertos conocimientos técnicos para su instalación y mantenimiento, y en proyectos de mayor tamaño puede presentar dificultades de rendimiento si no se configura correctamente.

\textbf{Ventajas:}
\begin{itemize}
    \item Open source y altamente personalizable.
    \item Gran comunidad de desarrolladores.
    \item Amplio catálogo de módulos y temas.
    \item Permite control total sobre el servidor y la base de datos.
\end{itemize}

\textbf{Desventajas:}
\begin{itemize}
    \item Necesita conocimientos técnicos para instalación y soporte.
    \item Requiere mantenimiento y actualizaciones manuales.
    \item Puede presentar problemas de rendimiento en tiendas grandes.
    \item Algunos módulos son de pago y elevan los costes.
\end{itemize}

\subsection{Comparación con la propuesta del TFG}
Las plataformas analizadas muestran distintas formas de crear una tienda online sin necesidad de conocimientos de programación. Wix y Squarespace destacan por su sencillez
y diseño intuitivo, aunque sacrifican flexibilidad y control sobre la aplicación. Por su parte, PrestaShop ofrece mayor capacidad de personalización, pero exige un nivel técnico más elevado y puede presentar limitaciones de rendimiento en proyectos de gran escala.
En contraste, la propuesta de este TFG plantea el desarrollo de una solución a medida  que permita un mayor control sobre la arquitectura, la seguridad y las integraciones, 
adaptándose específicamente a las necesidades del proyecto.


\begin{table}[H]
\centering
\begin{tabular}{|l|c|c|c|}
\hline
\textbf{Criterio} & \textbf{Wix} & \textbf{Squarespace} & \textbf{PrestaShop} \\ \hline
Facilidad de uso & Muy alta & Alta & Media \\ \hline
Personalización & Baja & Media & Alta \\ \hline
Escalabilidad & Limitada & Media & Alta \\ \hline
Requiere conocimientos técnicos & No & No & Sí \\ \hline
Coste & Medio & Medio & Variable \\ \hline
\end{tabular}
\caption{Comparativa de plataformas de creación de tiendas online}
\end{table}


\section{Tecnologías}

\subsection{Frameworks frontend}

\subsubsection{Angular}
Angular es un framework desarrollado por Google y basado en TypeScript \cite{angularDocs}. Ofrece una solución completa para el desarrollo de aplicaciones web a gran escala, con herramientas integradas como enrutamiento y gestión de formularios. Su principal desventaja es la necesidad de contar con conocimientos previos más avanzados, lo que incrementa la curva de aprendizaje.

\textbf{Ventajas:}
\begin{itemize}
    \item Framework completo con soporte oficial de Google.
    \item Tipado fuerte con TypeScript, que mejora la robustez del código.
    \item Soporte integrado para testing, formularios y enrutamiento.
    \item Comunidad amplia y documentación exhaustiva.
\end{itemize}

\textbf{Desventajas:}
\begin{itemize}
    \item Curva de aprendizaje pronunciada.
    \item Tamaño inicial del proyecto elevado.
    \item Requiere experiencia previa en TypeScript.
    \item Sintaxis más compleja frente a otras alternativas.
\end{itemize}

\subsubsection{React}
React es una tecnología desarrollada por Meta \cite{reactDocs}. Se trata de una biblioteca para la construcción de interfaces de usuario, que destaca por su flexibilidad y por el rendimiento que ofrece gracias al uso del Virtual DOM. Sin embargo, al no ser un framework completo, depende de librerías de la comunidad para cubrir funcionalidades adicionales o más avanzadas.

\textbf{Ventajas:}
\begin{itemize}
    \item Excelente rendimiento gracias al Virtual DOM.
    \item Gran comunidad y recursos de aprendizaje.
    \item Compatible con múltiples librerías y frameworks.
    \item Curva de aprendizaje accesible para principiantes.
\end{itemize}

\textbf{Desventajas:}
\begin{itemize}
    \item Necesidad de integrar herramientas externas (routing, estado).
    \item Menor estructura definida que Angular.
    \item Frecuentes actualizaciones y cambios en librerías.
    \item Requiere experiencia para mantener proyectos grandes.
\end{itemize}

\subsubsection{Vue}
Vue es un framework progresivo creado por Evan You, exingeniero de Google \cite{vueDocs}. Combina características de Angular y React, con un enfoque en la simplicidad y en la rapidez de aprendizaje. No obstante, su comunidad y ecosistema siguen siendo más reducidos en comparación con otras tecnologías consolidadas.

\textbf{Ventajas:}
\begin{itemize}
    \item Ligero y de fácil adopción.
    \item Documentación clara y organizada.
    \item Buena integración con proyectos existentes.
    \item Sintaxis sencilla y curva de aprendizaje rápida.
\end{itemize}

\textbf{Desventajas:}
\begin{itemize}
    \item Ecosistema más limitado que Angular o React.
    \item Menor presencia en grandes empresas.
    \item Menos recursos de formación avanzada.
    \item Riesgo de menor soporte a largo plazo.
\end{itemize}

\subsubsection{Comparación y elección}
Los tres frameworks permiten crear aplicaciones modernas, pero Angular ofrece una
estructura más completa desde el inicio. Para este TFG se ha optado por Angular, ya que
es la tecnología utilizada en Capgemini, empresa en la que voy a realizar las prácticas. De
esta forma, el proyecto servirá también como una preparación práctica, complementada
con cursos de formación, para no comenzar esa experiencia sin conocimientos previos.

\begin{table}[H]
\centering
\begin{tabular}{|l|c|c|c|}
\hline
\textbf{Criterio} & \textbf{Angular} & \textbf{React} & \textbf{Vue} \\ \hline
Lenguaje principal & TypeScript & JavaScript & JavaScript \\ \hline
Complejidad inicial & Alta & Media & Baja \\ \hline
Ecosistema & Muy amplio & Amplio & Moderado \\ \hline
Soporte empresarial & Alto & Alto & Medio \\ \hline
Rendimiento general & Alto & Muy alto & Alto \\ \hline
\end{tabular}
\caption{Comparativa de frameworks frontend}
\end{table}


\subsection{Frameworks backend}

\subsubsection{Spring Boot}
Spring Boot es un framework del ecosistema Java que facilita la creación de aplicaciones backend robustas y escalables \cite{springDocs}. Entre sus principales ventajas se encuentran su amplia comunidad, la integración con múltiples librerías y la facilidad para configurar proyectos complejos. Como inconvenientes, requiere conocimientos previos en Java y, según el tamaño de la aplicación, puede resultar más pesado que otras alternativas.

\textbf{Ventajas:}
\begin{itemize}
    \item Framework maduro con gran soporte empresarial.
    \item Integración nativa con bases de datos y APIs REST.
    \item Facilita pruebas y despliegues automáticos.
    \item Gran estabilidad y seguridad.
\end{itemize}

\textbf{Desventajas:}
\begin{itemize}
    \item Requiere conocimientos avanzados en Java.
    \item Consumo de recursos más alto que Node.js.
    \item Curva de aprendizaje larga.
    \item Configuraciones iniciales complejas.
\end{itemize}

\subsubsection{Node.js}
Node.js es un entorno de ejecución de JavaScript que permite utilizar este lenguaje en la configuración y desarrollo de servidores \cite{nodeDocs}. Destaca por su rendimiento en aplicaciones en tiempo real y por la ventaja de compartir el mismo lenguaje en frontend y backend, lo que agiliza el desarrollo. Su principal limitación es que no está tan orientado a aplicaciones de gran escala como otros frameworks más consolidados.

\textbf{Ventajas:}
\begin{itemize}
    \item Excelente rendimiento en tiempo real.
    \item Usa el mismo lenguaje en cliente y servidor.
    \item Amplia comunidad y soporte.
    \item Ecosistema extenso con NPM.
\end{itemize}

\textbf{Desventajas:}
\begin{itemize}
    \item Menor rendimiento en tareas muy pesadas.
    \item Dificultad para proyectos altamente estructurados.
    \item Depende mucho de librerías externas.
    \item Escalabilidad más limitada que Spring Boot.
\end{itemize}

\subsubsection{Django}
Django es un framework de Python que sigue el principio "batteries included", lo que significa que incorpora numerosas funcionalidades listas para usar \cite{djangoDocs}. Es adecuado para el desarrollo rápido de prototipos y proyectos de tamaño medio, con una curva de aprendizaje moderada. Sin embargo, su rendimiento puede verse limitado en aplicaciones muy exigentes.

\textbf{Ventajas:}
\begin{itemize}
    \item Gran cantidad de herramientas integradas.
    \item Excelente para desarrollo rápido.
    \item Seguridad y control de autenticación robusto.
    \item Documentación muy completa.
\end{itemize}

\textbf{Desventajas:}
\begin{itemize}
    \item Menor rendimiento en aplicaciones grandes.
    \item Menor integración con sistemas Java.
    \item No tan extendido en entornos empresariales.
    \item Difícil de optimizar para proyectos complejos.
\end{itemize}

\subsubsection{Comparación y elección}
Spring Boot, Node.js y Django son alternativas viables para el desarrollo de un backend. Node.js destaca por su rapidez en tiempo real y Django por la simplicidad de Python, pero para este TFG se ha optado por Spring Boot. Esta elección responde a que es la tecnología utilizada en Capgemini, empresa en la que realizaré las prácticas, lo
que me permite formarme previamente mediante cursos y adquirir los conceptos básicos necesarios. Además, Spring Boot destaca por su robustez, su madurez en entornos empresariales y porque complementa de forma óptima la arquitectura planteada con Angular y SQL.

\begin{table}[H]
\centering
\begin{tabular}{|l|c|c|c|}
\hline
\textbf{Criterio} & \textbf{Spring Boot} & \textbf{Node.js} & \textbf{Django} \\ \hline
Lenguaje base & Java & JavaScript & Python \\ \hline
Escalabilidad & Alta & Media & Media \\ \hline
Curva de aprendizaje & Alta & Media & Baja \\ \hline
Rendimiento general & Alto & Alto & Medio \\ \hline
Soporte empresarial & Muy alto & Medio & Medio \\ \hline
\end{tabular}
\caption{Comparativa de frameworks backend}
\end{table}


\subsection{Agentes de automatización}

\subsubsection{Zapier}
Zapier es una herramienta para automatizar tareas mediante la conexión de aplicaciones. Destaca por su simplicidad y facilidad de uso, aunque sus funcionalidades avanzadas requieren una suscripción de pago\cite{zapierDocs}.

\textbf{Ventajas:}
\begin{itemize}
    \item Muy fácil de usar.
    \item Gran catálogo de aplicaciones compatibles.
    \item Ideal para tareas simples y rápidas.
    \item Interfaz visual clara y moderna.
\end{itemize}

\textbf{Desventajas:}
\begin{itemize}
    \item Las funciones avanzadas requieren suscripción.
    \item Limitado control sobre la configuración interna.
    \item Escasa flexibilidad para integraciones complejas.
    \item No permite autoalojamiento.
\end{itemize}

\subsubsection{Make}
Make, anteriormente conocido como Integromat, es una herramienta que ofrece un
sistema visual para crear flujos de trabajo más complejos que en Zapier. Es potente y
flexible, aunque su uso intensivo también depende de planes de pago y de la integración
con servicios externos\cite{makeDocs}.

\textbf{Ventajas:}
\begin{itemize}
    \item Muy flexible para automatizaciones avanzadas.
    \item Interfaz visual potente y configurable.
    \item Permite múltiples pasos por flujo.
    \item Buen equilibrio entre facilidad y personalización.
\end{itemize}

\textbf{Desventajas:}
\begin{itemize}
    \item Algunas funciones requieren plan de pago.
    \item Requiere conexión constante a la nube.
    \item Mayor complejidad de uso frente a Zapier.
    \item Menor documentación en español.
\end{itemize}

\subsubsection{n8n}
n8n es una plataforma de automatización de código abierto que permite crear flujos personalizados e instalarlos en servidores propios, sin necesidad de planes de pago. Ofrece un gran control sobre la configuración y resulta una herramienta intuitiva y fácil de utilizar \cite{n8nDocs}.

\textbf{Ventajas:}
\begin{itemize}
    \item Totalmente open source y gratuita.
    \item Puede instalarse en servidores propios.
    \item Gran control sobre la configuración.
    \item Compatible con múltiples servicios y APIs.
\end{itemize}

\textbf{Desventajas:}
\begin{itemize}
    \item Requiere configuración inicial manual.
    \item Menor comunidad que Zapier.
    \item No tan intuitiva para principiantes.
    \item Necesita recursos del servidor donde se aloje.
\end{itemize}

\subsubsection{Comparación y elección}
Zapier y Make destacan por su facilidad de uso y popularidad, pero dependen de planes de suscripción y de servicios externos, además de no ofrecer un control total sobre la configuración. En este TFG se ha optado por n8n por ser una solución open source, gratuita y altamente configurable, lo que facilita su integración con la arquitectura  propuesta.

\begin{table}[H]
\centering
\begin{tabular}{|l|c|c|c|}
\hline
\textbf{Criterio} & \textbf{Zapier} & \textbf{Make} & \textbf{n8n} \\ \hline
Tipo de licencia & De pago & Freemium & Open source \\ \hline
Facilidad de uso & Muy alta & Alta & Media \\ \hline
Control y personalización & Bajo & Medio & Alto \\ \hline
Coste & Elevado & Medio & Gratuito \\ \hline
Soporte técnico & Alto & Medio & Medio \\ \hline
\end{tabular}
\caption{Comparativa de agentes de automatización}
\end{table}


\subsection{Bases de datos}

\subsubsection{SQL}
Las bases de datos relacionales, como MySQL o PostgreSQL, utilizan un modelo basado en tablas y relaciones entre ellas \cite{mysqlDocs}. Destacan por su consistencia, su alto nivel de seguridad y por el uso del lenguaje SQL, que es un estándar ampliamente adoptado en el entorno empresarial.

\subsubsection{NoSQL}
Las bases de datos NoSQL, como MongoDB, almacenan la información en estructuras más flexibles, como documentos o grafos. Ofrecen mayor escalabilidad y rapidez en ciertas operaciones, aunque sacrifican consistencia y mecanismos de seguridad más avanzados. \cite{mongodbDocs}.

\subsubsection{Comparación y elección}
Para este TFG se ha optado por utilizar \textbf{MySQL} como sistema de gestión de bases de datos. La elección se debe a su estabilidad, su integración nativa con Spring Boot y su capacidad para garantizar la seguridad de los datos. Además, a lo largo de la carrera se ha trabajado ampliamente con bases de datos SQL, mientras que NoSQL apenas se ha tratado, por lo que MySQL resulta una opción más familiar y adecuada para el desarrollo del proyecto.

\begin{table}[H]
\centering
\begin{tabular}{|l|c|c|}
\hline
\textbf{Criterio} & \textbf{SQL (MySQL)} & \textbf{NoSQL (MongoDB)} \\ \hline
Estructura & Tablas relacionales & Documentos / grafos \\ \hline
Consistencia & Alta & Media \\ \hline
Seguridad & Alta & Media \\ \hline
Escalabilidad & Media & Alta \\ \hline
Soporte empresarial & Muy alto & Medio \\ \hline
\end{tabular}
\caption{Comparativa entre bases de datos SQL y NoSQL}
\end{table}

\subsection{Pasarelas de pago}

\subsubsection{PayPal}
PayPal es una de las plataformas de pago online más conocidas y utilizadas a nivel mundial \cite{paypalDocs}. Permite realizar transacciones de forma sencilla y segura, tanto con tarjeta como con saldo en cuenta. Es muy popular entre los usuarios por su rapidez y confianza, aunque cobra comisiones algo más altas que otras alternativas.

\textbf{Ventajas:}
\begin{itemize}
    \item Gran confianza y reconocimiento internacional.
    \item Permite pagar sin necesidad de introducir datos bancarios en cada compra.
    \item Interfaz sencilla para el usuario.
    \item Compatible con muchas plataformas de comercio electrónico.
\end{itemize}

\textbf{Desventajas:}
\begin{itemize}
    \item Comisiones elevadas por transacción.
    \item Poca flexibilidad para personalizar el flujo de pago.
    \item Integración técnica menos avanzada que otras opciones.
    \item Posibles retenciones temporales de fondos en algunas operaciones.
\end{itemize}


\subsubsection{Redsys}
Redsys es la pasarela de pago más extendida en España y se utiliza en la mayoría de tiendas online que trabajan con bancos nacionales \cite{redsysDocs}. Ofrece un alto nivel de seguridad y cumple con las normativas europeas, aunque su configuración inicial puede ser más compleja y su documentación menos accesible para desarrolladores.

\textbf{Ventajas:}
\begin{itemize}
    \item Muy utilizada en España, con soporte por parte de los principales bancos.
    \item Alta seguridad gracias al uso de sistemas 3D Secure.
    \item Cumple con las normativas PSD2 y SCA.
    \item Fiable y con buena reputación en entornos comerciales.
\end{itemize}

\textbf{Desventajas:}
\begin{itemize}
    \item Configuración inicial compleja.
    \item Documentación técnica limitada.
    \item Poco orientada a desarrolladores.
    \item Limitada flexibilidad en integraciones personalizadas.
\end{itemize}


\subsubsection{Stripe}
Stripe es una pasarela de pago moderna pensada especialmente para desarrolladores \cite{stripeDocs}. Su API es clara y bien documentada, lo que facilita mucho la integración con aplicaciones web. Permite gestionar pagos con tarjeta, transferencias y otros métodos de forma segura y rápida. Es una opción muy utilizada por startups y empresas tecnológicas por su facilidad de uso y sus amplias posibilidades de personalización.

\textbf{Ventajas:}
\begin{itemize}
    \item API muy completa y fácil de integrar.
    \item Admite múltiples métodos de pago y divisas.
    \item Buen equilibrio entre seguridad y personalización.
    \item Excelente documentación y comunidad activa.
\end{itemize}

\textbf{Desventajas:}
\begin{itemize}
    \item Requiere conocimientos técnicos básicos para la configuración.
    \item Comisiones similares a las de PayPal.
    \item Depende de una conexión estable con la API.
    \item No disponible en todos los países.
\end{itemize}


\subsubsection{Comparación y elección}
PayPal, Redsys y Stripe son tres opciones válidas para integrar pagos en una tienda online. PayPal destaca por su popularidad y facilidad de uso, mientras que Redsys es la alternativa más común en España y ofrece gran seguridad. Sin embargo, para este TFG se ha optado por \textbf{Stripe} debido a su flexibilidad, buena documentación y facilidad de integración con tecnologías modernas como Spring Boot y Angular. Además, su API REST simplifica la conexión con el backend y garantiza un proceso de pago seguro y fluido para el usuario.

\begin{table}[H]
\centering
\begin{tabular}{|l|c|c|c|}
\hline
\textbf{Criterio} & \textbf{PayPal} & \textbf{Redsys} & \textbf{Stripe} \\ \hline
Popularidad & Muy alta & Alta & Alta \\ \hline
Facilidad de integración & Media & Baja & Alta \\ \hline
Comisiones & Altas & Medias & Medias \\ \hline
Orientación a desarrolladores & Baja & Baja & Muy alta \\ \hline
Seguridad & Alta & Muy alta & Alta \\ \hline
\end{tabular}
\caption{Comparativa de pasarelas de pago}
\end{table}




\chapter{Requisitos, especificaciones, coste, riesgos, viabilidad}
% Contenidos del capítulo
% Las secciones presentadas son orientativas y no representan
% necesariamente la organización que debe tener este capítulo.
% =============================
% Capítulo 3: Requisitos, especificaciones, coste, riesgos y viabilidad
% =============================
En este capítulo se presenta un estudio inicial del proyecto, analizando los requisitos, especificaciones, costes, riesgos y viabilidad del sistema desarrollado. 
El objetivo es definir las \textbf{funcionalidades necesarias} para el correcto funcionamiento de la aplicación, detallar sus \textbf{especificaciones técnicas}, estimar el \textbf{coste y el tiempo de desarrollo}, identificar los \textbf{principales riesgos} y, finalmente, valorar la \textbf{viabilidad global} del proyecto. A continuación, se detallan los distintos apartados que conforman este estudio.

% Requisitos del sistema
\section{Requisitos}

El sistema propuesto consiste en una \textbf{plataforma web de comercio electrónico para una droguería} que permite a los usuarios registrarse, explorar productos, filtrar resultados, gestionar su carrito de compra, realizar pagos mediante un servicio externo (Stripe) y recibir notificaciones automáticas mediante el sistema n8n. 
Por parte del administrador, el sistema debe permitir filtrar pedidos, añadir productos, así como consultar y modificar los pedidos existentes.

A continuación, se detallan los requisitos funcionales y no funcionales definidos para el proyecto.

\subsection{Requisitos funcionales}

\begin{itemize}
    \item \textbf{RF-01:} El sistema debe permitir el registro y autenticación de usuarios mediante correo electrónico y contraseña.
    \item \textbf{RF-02:} El sistema debe permitir que los usuarios consulten y filtren el catálogo de productos disponible.
    \item \textbf{RF-03:} El sistema debe permitir añadir, modificar o eliminar productos del carrito.
    \item \textbf{RF-04:} El sistema debe permitir realizar pedidos a través de una pasarela de pago (Stripe).
    \item \textbf{RF-05:} El sistema debe permitir al administrador gestionar el inventario de productos (crear, editar, eliminar).
    \item \textbf{RF-06:} El sistema debe generar y almacenar facturas de los pedidos realizados.
    \item \textbf{RF-07:} El sistema debe enviar notificaciones automáticas de confirmación de pedido y aviso de bajo stock mediante n8n.
    \item \textbf{RF-08:} El sistema debe permitir consultar el historial de pedidos y su estado.
    \item \textbf{RF-09:} El sistema debe eliminar automáticamente los carritos inactivos tras un periodo determinado.
\end{itemize}

\subsection{Requisitos no funcionales}

\begin{itemize}
    \item \textbf{RNF-01:} El sistema debe desarrollarse siguiendo una arquitectura cliente-servidor (Angular + Spring Boot + MySQL).
    \item \textbf{RNF-02:} La interfaz debe ser intuitiva, responsiva y accesible desde distintos dispositivos.
    \item \textbf{RNF-03:} El sistema debe garantizar la seguridad de los datos mediante cifrado de contraseñas y uso de HTTPS.
    \item \textbf{RNF-04:} La base de datos debe permitir transacciones seguras y mantener la integridad referencial.
    \item \textbf{RNF-05:} El código debe estar estructurado siguiendo buenas prácticas de desarrollo (modularización, uso de control de versiones).
    \item \textbf{RNF-06:} El sistema debe permitir su despliegue local o en la nube sin necesidad de licencias adicionales.
    \item \textbf{RNF-07:} Las automatizaciones mediante n8n deben ser ejecutadas en un entorno controlado y seguro.
\end{itemize}

% Especificación del sistema a partir de lo recogido en los requisitos
\section{Especificaciones}

El sistema está diseñado como una \textbf{aplicación web moderna} basada en una arquitectura \textbf{frontend-backend}, en la que el cliente (Angular) se comunica con el servidor (Spring Boot) mediante \textit{API REST}. 
La base de datos utilizada es \textbf{MySQL}, encargada de almacenar toda la información relativa a usuarios, productos, pedidos, pagos y facturas. 
Además, el sistema incorpora \textbf{n8n} como herramienta de automatización para gestionar tareas recurrentes, como el envío de correos postcompra o las notificaciones de bajo stock.

El sistema distingue entre dos tipos principales de usuarios:

\begin{itemize}
    \item \textbf{Cliente:} puede registrarse, gestionar su carrito, realizar compras y consultar su historial de pedidos.
    \item \textbf{Administrador:} gestiona el catálogo de productos, controla los pedidos y supervisa las automatizaciones.
\end{itemize}

La aplicación está desarrollada íntegramente en \textbf{español}, es \textbf{multiplataforma} y accesible desde cualquier navegador moderno. 
Todos los componentes empleados son \textbf{gratuitos y de código abierto}, lo que facilita su instalación y mantenimiento sin coste adicional.

% Costes temporales y económicos
\section{Tareas y costes}

Para planificar el desarrollo del proyecto se ha elaborado una \textbf{Estructura de Descomposición del Trabajo (EDT)} que organiza las actividades en cinco fases principales: estudio, análisis, diseño, implementación y pruebas. 
Esta estructura permite identificar las dependencias entre tareas y establecer una secuencia lógica de ejecución.

\subsection{Definición de tareas}

\begin{table}[H]
\centering
\begin{tabular}{|c|p{9cm}|c|}
\hline
\textbf{ID} & \textbf{Nombre de la tarea} & \textbf{Dependencias} \\
\hline
1 & Estudio y especificación del proyecto & -- \\
1.1 & Definición de requisitos & -- \\
1.2 & Estimación de costes y planificación & 1.1 \\
1.3 & Análisis de riesgos & 1.2 \\
1.4 & Estudio de herramientas (Angular, Spring Boot, n8n) & 1.3 \\
2 & Análisis del sistema & 1 \\
2.1 & Casos de uso de la aplicación & 1.4 \\
2.2 & Modelo de datos y entidades principales & 2.1 \\
3 & Diseño del sistema & 2 \\
3.1 & Arquitectura general (frontend-backend-n8n) & 2.2 \\
3.2 & Diagramas UML y diseño de base de datos & 3.1 \\
4 & Implementación & 3 \\
4.1 & Desarrollo del backend en Spring Boot & 3.2 \\
4.2 & Desarrollo del frontend en Angular & 4.1 \\
4.3 & Integración con Stripe y MySQL & 4.2 \\
4.4 & Configuración de flujos n8n & 4.3 \\
5 & Pruebas y validación & 4 \\
5.1 & Pruebas unitarias e integración & 4.4 \\
5.2 & Validación con usuarios y optimización final & 5.1 \\
\hline
\end{tabular}
\caption{Estructura de Descomposición del Trabajo (EDT) y dependencias entre tareas.}
\end{table}
\clearpage  

\subsection{Estimación temporal}

Para estimar la duración de cada fase se ha utilizado la técnica \textbf{PERT (Program Evaluation and Review Technique)}, 
que permite obtener una estimación temporal basada en tres valores: el tiempo optimista, el más probable y el pesimista. 
Esta metodología se fundamenta en los principios clásicos de estimación temporal descritos en la gestión de proyectos \cite{PMBOK2021} y 
calcula el tiempo estimado (TE) mediante la fórmula:

\[
TE = \frac{O + 4M + P}{6}
\]

donde \textit{O} es el tiempo optimista, \textit{M} el más probable y \textit{P} el pesimista.


\begin{table}[H]
\centering
\begin{tabular}{|l|c|c|c|c|}
\hline
\textbf{Fase} & \textbf{O} & \textbf{M} & \textbf{P} & \textbf{TE (días)} \\
\hline
Estudio y especificación & 3 & 5 & 7 & 5 \\
Análisis & 4 & 6 & 8 & 6 \\
Diseño & 6 & 8 & 10 & 8 \\
Implementación & 20 & 25 & 30 & 25 \\
Pruebas & 6 & 8 & 10 & 8 \\
\hline
\textbf{Total estimado} &  &  &  & \textbf{52 días} \\
\hline
\end{tabular}
\caption{Estimación temporal de las fases del proyecto mediante el método PERT.}
\end{table}

Además de la técnica \textbf{PERT}, se ha aplicado el método de \textbf{juicio de expertos} con el objetivo de validar las estimaciones obtenidas y asegurar que los tiempos calculados sean realistas. 
Para ello, se han considerado las opiniones de tres perfiles con diferentes niveles de experiencia en el ámbito del desarrollo de software, lo que permitió ajustar las duraciones de las tareas y obtener una planificación más equilibrada:

\begin{itemize}
    \item \textbf{Perfil 1 — Estudiante avanzado:} estudiante de último curso del Grado en Ingeniería Informática, con experiencia en proyectos académicos y conocimientos de programación web. 
    Estimó que el proyecto podría completarse en torno a unos \textbf{55 días}, dedicando más tiempo a las fases de diseño e implementación, al considerar que estas suponen un mayor reto técnico en un proyecto individual.

    \item \textbf{Perfil 2 — Desarrollador junior:} profesional con unos dos años de experiencia en el desarrollo de aplicaciones web. 
    Según su criterio, la duración total estaría alrededor de \textbf{50 días}, al poder abordar las fases de análisis e implementación con mayor agilidad, aunque manteniendo márgenes prudentes en las fases de pruebas y documentación.

    \item \textbf{Perfil 3 — Desarrollador senior/jefe de proyecto:} profesional con más de cinco años de experiencia en la gestión y supervisión de proyectos informáticos. 
    Consideró que el proyecto podría completarse en torno a \textbf{48 días}, optimizando las fases de diseño e implementación gracias a una mejor planificación y experiencia previa en proyectos similares. Su punto de vista permitió validar la coherencia general del calendario y la estimación global del esfuerzo.
\end{itemize}
\clearpage
El contraste de opiniones entre estos perfiles permitió confirmar que las estimaciones iniciales eran adecuadas, situándose todas en un rango comprendido entre los 48 y 55 días, con una media aproximada de \textbf{50 días}. 
A partir de sus aportaciones, se fijó una duración final de \textbf{52 días}, incorporando un pequeño margen de contingencia que refleja un equilibrio entre las distintas perspectivas.

\subsection{Diagrama de Gantt}

El siguiente diagrama de Gantt muestra la planificación temporal del proyecto de manera visual, 
indicando la duración de cada fase y su relación con las tareas definidas en la estructura de descomposición del trabajo (EDT). 
Se ha generado a partir de las estimaciones realizadas mediante la técnica PERT y validadas con el juicio de expertos.

\begin{sidewaysfigure}
\centering
\includegraphics[width=\textwidth]{figs/GanttTodo.pdf}
\caption{Diagrama de Gantt del proyecto (01/10/2025–07/01/2026).}
\end{sidewaysfigure}

\clearpage
\subsection{Estimación de costes}

Para el cálculo de costes se han tenido en cuenta tanto los recursos humanos como los materiales y el software utilizado. 
Aunque el proyecto ha sido desarrollado por una sola persona, se ha desglosado el trabajo según distintos perfiles habituales en el desarrollo de software: 
jefe de proyecto, analista, desarrollador y tester. 
De esta forma se obtiene una estimación más ajustada y realista del esfuerzo requerido.

\begin{table}[H]
\centering
\begin{tabular}{|L{4.5cm}|c|}
\hline
\textbf{Perfil} & \textbf{Horas asignadas} \\
\hline
Jefe de proyecto / Analista & 40 h \\
Desarrollador backend & 140 h \\
Desarrollador frontend & 100 h \\
Tester / Validación & 40 h \\
\hline
\textbf{Total de horas estimadas} & \textbf{320 h} \\
\hline
\end{tabular}
\caption{Reparto de horas por perfil utilizado para la estimación de costes.}
\end{table}

\begin{table}[H]
\centering
\begin{tabular}{|L{4.5cm}|L{7.5cm}|c|}
\hline
\textbf{Concepto} & \textbf{Detalle} & \textbf{Coste estimado} \\
\hline
\textbf{Coste de personal} & Jefe de proyecto / analista: 40 h × 18 €/h & 720 € \\
 & Desarrollador backend: 140 h × 12 €/h & 1.680 € \\
 & Desarrollador frontend: 100 h × 12 €/h & 1.200 € \\
 & Tester / validación: 40 h × 10 €/h & 400 € \\
\hline
\textbf{Subtotal personal} & & \textbf{4.000 €} \\
\hline
\textbf{Seguridad Social (30 \%)} & Aplicado al coste de personal (30 \% × 4.000 €) & \textbf{1.200 €} \\
\hline
\textbf{Coste de software} & Herramientas gratuitas (Angular, Spring Boot, MySQL, n8n, VSCode, Postman) & 0 € \\
\hline
\textbf{Coste de software con licencia universitaria} & Visual Paradigm (licencia académica proporcionada por la universidad) & 0 € \\
\hline
\textbf{Coste de hardware (amortizado)} & Portátil personal (1.200 € de valor, amortizado 4 meses sobre 4 años) & 100 € \\
\hline
\textbf{Costes indirectos} & Electricidad, Internet y agua durante el desarrollo (estimado) & 80 € \\
\hline
\textbf{Total estimado del proyecto} & & \textbf{5.380 €} \\
\hline
\end{tabular}
\caption{Estimación de costes detallada por perfil y recursos.}
\vspace{0.3cm}
\textit{Nota: el valor del hardware se ha amortizado en base a una vida útil estimada de cuatro años.}
\end{table}
\clearpage
Además, se incluyen las características del equipo utilizado durante el desarrollo, ya que impactan tanto en la amortización del hardware como en el rendimiento del proceso de trabajo:

\begin{itemize}
    \item \textbf{CPU:} Intel Core i7-12700H
    \item \textbf{RAM:} 16 GB DDR5
    \item \textbf{Almacenamiento:} SSD NVMe 1 TB
    \item \textbf{Gráfica:} NVIDIA RTX 3050 Ti
    \item \textbf{Sistema operativo:} Windows 11
\end{itemize}

Para la planificación temporal se ha aplicado la técnica \textbf{PERT}, combinada con el \textbf{juicio de expertos}, 
comparando los valores calculados con estimaciones de proyectos similares y con referencias salariales obtenidas de portales como Indeed \cite{Indeed2025}. 
Además, se consultaron tres perfiles con diferente nivel de experiencia en el desarrollo de software, lo que permitió contrastar las horas estimadas y ajustar los tiempos de las tareas.

Finalmente, el coste total del proyecto se obtiene mediante la siguiente expresión:

\[
C_{\text{total}} = C_{\text{personal}} + C_{\text{SeguridadSocial}} + C_{\text{hardware}} + C_{\text{software}} + C_{\text{indirectos}}
\]

\[
C_{\text{total}} = 4\,000 + 1\,200 + 100 + 0 + 80 = \textbf{5\,380\,€}
\]

El resultado confirma que el proyecto mantiene un coste razonable y acorde a su nivel de complejidad. 
El uso de herramientas libres y de una infraestructura propia reduce significativamente el coste total, 
manteniendo una buena relación entre esfuerzo, recursos y resultados obtenidos.


% Riesgos que pueden incurrir en el desarrollo del sistema
\clearpage
\section{Riesgos}

La gestión de riesgos permite anticipar posibles incidencias y definir estrategias preventivas que minimicen su impacto en el desarrollo del proyecto. 
En la siguiente tabla se resumen los principales riesgos identificados junto con su probabilidad, impacto y categoría general.

\begin{table}[H]
\centering
\begin{tabular}{|L{6.5cm}|c|c|c|}
\hline
\textbf{Riesgo identificado} & \textbf{Probabilidad} & \textbf{Impacto} & \textbf{Categoría} \\
\hline
Problemas de integración entre módulos Angular–Spring Boot & Media & Alta & Técnica \\
\hline
Fallos en la pasarela de pago (Stripe) & Baja & Alta & Técnica \\
\hline
Cambios en los requisitos durante el desarrollo & Media & Media & Gestión \\
\hline
Sobrecarga de trabajo o retrasos en la planificación & Media & Media & Organización \\
\hline
Fallos de seguridad en la API o la base de datos & Baja & Alta & Seguridad \\
\hline
Falta de experiencia inicial con n8n & Media & Media & Técnica \\
\hline
\end{tabular}
\caption{Principales riesgos identificados durante el desarrollo del proyecto.}
\end{table}

A continuación, se describen las medidas generales de \textbf{mitigación y contingencia} aplicadas a los riesgos identificados:

\begin{itemize}
    \item \textbf{Problemas de integración entre módulos Angular–Spring Boot:} se intentará detectar pronto cualquier error realizando pruebas cada vez que se conecten nuevas partes del sistema. También se mantendrá una buena organización del código para facilitar la comunicación entre ambos módulos.
    
    \item \textbf{Fallos en la pasarela de pago (Stripe):} antes de activar los pagos reales, se harán pruebas en el modo de simulación que ofrece la plataforma. En caso de error, se podrá reintentar el pago o registrar el pedido para revisarlo después.
    
    \item \textbf{Cambios en los requisitos durante el desarrollo:} se dejará algo de margen en la planificación para poder adaptarse a posibles cambios sin afectar al trabajo principal. Además, se priorizarán las funciones más importantes para tener siempre una versión funcional.
    
    \item \textbf{Sobrecarga de trabajo o retrasos en la planificación:} se dividirá el trabajo en tareas pequeñas y se intentará cumplir objetivos semanales. También se reservará algo de tiempo extra para posibles imprevistos o revisiones de última hora.
    
    \item \textbf{Fallos de seguridad en la aplicación o base de datos:} se harán copias de seguridad con frecuencia y se guardarán las contraseñas de forma segura. Además, se mantendrán las herramientas actualizadas para evitar problemas.
    
    \item \textbf{Falta de experiencia con n8n:} se dedicará tiempo a aprender su funcionamiento antes de integrarlo por completo. Se probarán ejemplos sencillos para entender cómo crear los flujos y se consultará la documentación cuando sea necesario.
\end{itemize}
Gracias a estas medidas, se espera que los riesgos identificados tengan un impacto limitado en el desarrollo y no comprometan los objetivos principales del proyecto.



% Viabilidad del proyecto presentado
\section{Viabilidad}

\subsection{Viabilidad legal}

El proyecto no presenta impedimentos legales, ya que no maneja datos personales sensibles y cumple con la normativa vigente, especialmente con el Reglamento General de Protección de Datos (RGPD). 
Los pagos se realizan mediante la plataforma \textbf{Stripe}, que se encarga de procesar la información de forma segura y cifrada, evitando que el sistema tenga que almacenar datos de tarjetas. 
Todas las tecnologías empleadas son de \textbf{código abierto} o disponen de \textbf{licencias académicas gratuitas}, por lo que su uso es totalmente legal en el contexto de un proyecto universitario.

\subsection{Viabilidad técnica}

El uso conjunto de \textbf{Angular}, \textbf{Spring Boot}, \textbf{MySQL} y \textbf{n8n} proporciona una solución estable, escalable y fácil de mantener. 
Son herramientas actuales, bien documentadas y con una amplia comunidad de desarrolladores, lo que facilita la resolución de problemas y reduce el riesgo de incompatibilidades. 
Además, el diseño modular del sistema permite incorporar futuras mejoras, como nuevos métodos de pago o automatizaciones adicionales, sin necesidad de modificar la estructura principal del proyecto.

\subsection{Viabilidad económica}

El proyecto resulta \textbf{económicamente viable}, ya que su desarrollo no requiere inversión en licencias ni en infraestructura adicional. 
El software utilizado es \textbf{gratuito y de código abierto}, y el equipo empleado para el desarrollo es personal, por lo que los únicos costes reales corresponden al tiempo de dedicación y al uso del propio ordenador. 
Teniendo en cuenta que el coste total estimado del proyecto es de aproximadamente \textbf{4.400 €}, se puede concluir que se trata de una solución accesible, eficiente y sostenible para un entorno académico o de pequeña empresa.


\chapter{Análisis}
% Contenidos del capítulo.
% Las secciones presentadas son orientativas y no representan
% necesariamente la organización que debe tener este capítulo.

\section{Diagramas de análisis}

\subsection{Diagrama de casos de uso}

Para representar las interacciones entre los diferentes actores del sistema se ha elaborado el diagrama de casos de uso mostrado en la Figura~\ref{fig:casos_uso}. 
Este permite visualizar de forma general las funcionalidades principales y la relación entre el cliente, el administrador y el agente automatizador \textit{n8n}.

\begin{figure}[H]
    \centering
    \includegraphics[height=1\textwidth, width=1.05\textwidth, keepaspectratio]{figs/Caso_de uso_def.png}
    \caption{Diagrama de casos de uso del sistema.}
    \label{fig:casos_uso}
\end{figure}
\clearpage

\subsection{Diagrama de clases de análisis}

En la Figura~\ref{fig:clases_analisis} se muestra el \textbf{diagrama de clases de análisis} del sistema. 
Este modelo conceptual identifica las principales entidades que intervienen en el dominio de la aplicación, 
así como las relaciones que existen entre ellas. 
El cliente dispone de un carrito y puede realizar múltiples pedidos, 
cada uno de los cuales está asociado a un pago y a varios productos. 
Además, el cliente puede almacenar varias direcciones de envío, 
y los productos se agrupan dentro de categorías. 
El diagrama abstrae los detalles técnicos y se centra únicamente en los conceptos del negocio, 
sirviendo como base para el posterior diagrama de clases de diseño.

\begin{figure}[H]
    \centering
    \includegraphics[width=0.85\textwidth]{figs/CasosdeUsoAnalisis.png}
    \caption{Diagrama de clases de análisis del sistema.}
    \label{fig:clases_analisis}
\end{figure}
\clearpage

\subsection{Diagrama de secuencia general del sistema (DSGS)}

La Figura~\ref{fig:dsgs} muestra el \textbf{diagrama de secuencia general del sistema (DSGS)}, 
que representa de forma dinámica las interacciones entre los principales actores externos 
y la \textit{WebApp}. En él se detalla el flujo completo de una compra, 
desde el inicio de sesión del cliente, la selección de productos y la realización del pedido, 
hasta la comunicación con la pasarela de pago \textit{Stripe} 
y la notificación del resultado a la plataforma de automatización \textit{n8n}. 
El diagrama también incluye un flujo alternativo en el que se muestra el manejo de un posible error de pago. 
Este modelo complementa al diagrama de casos de uso al ofrecer una visión temporal del comportamiento global del sistema.

\begin{figure}[H]
    \centering
    \includegraphics[width=0.9\textwidth]{figs/DSGS.png}
    \caption{Diagrama de secuencia general del sistema (DSGS).}
    \label{fig:dsgs}
\end{figure}

\clearpage





\chapter{Diseño}
% Contenidos del capítulo.
% Las secciones presentadas son orientativas y no representan
% necesariamente la organización que debe tener este capítulo.

% Diagramas de clases, de secuencia, de despliegue, diseño de
% pantallas, etc


En este capítulo se presenta el proceso de diseño del sistema desarrollado, que abarca tanto la definición de la arquitectura general como los distintos diagramas UML que describen la estructura y el comportamiento de la aplicación. Finalmente, se incluye el diseño de la base de datos, que servirá como soporte a la implementación posterior.

\section{Arquitectura general del sistema}

El sistema se basa en una arquitectura de tres capas que facilita la separación de responsabilidades y la escalabilidad del proyecto:

\begin{itemize}
    \item \textbf{Frontend:} desarrollado con un framework moderno Angular, encargado de la interacción con el usuario y la comunicación con el backend mediante peticiones HTTP.
    \item \textbf{Backend:} implementado con \textit{Spring Boot}, que gestiona la lógica de negocio, las operaciones CRUD y la conexión con la base de datos.
    \item \textbf{Base de datos:} gestionada con el motor \textit{MySQL}, donde se almacenan las entidades principales del sistema.
\end{itemize}

Además, el sistema integra un agente automatizador \textit{n8n} encargado de tareas como el envío de notificaciones post-compra, la generación automática de facturas o la detección de carritos abandonados.  
Esta arquitectura modular permite una comunicación fluida entre componentes y facilita la futura ampliación del sistema.

\section{Diagramas UML}

Los diagramas UML se han elaborado con la herramienta \textit{Visual Paradigm} para representar de forma visual los componentes y sus interacciones. Se incluyen diagramas de casos de uso, clases y secuencia.

\subsection{Diagrama de clases}

En la Figura~\ref{fig:clases} se muestra el diagrama de clases que define la estructura interna del sistema, las entidades principales y sus relaciones.  
Entre las clases más destacadas se encuentran:

\begin{itemize}
    \item \textbf{Usuario:} almacena la información del cliente o administrador.
    \item \textbf{Producto:} contiene los datos de los artículos disponibles.
    \item \textbf{Carrito} y \textbf{CarritoItems:} gestionan los productos añadidos por cada usuario.
    \item \textbf{Pedido}, \textbf{Pago} y \textbf{Factura:} representan el flujo de compra y facturación.
\end{itemize}

\begin{figure}[H]
    \centering
    \includegraphics[height=1\textwidth, width=1.05\textwidth, keepaspectratio]{figs/clasesdef.png}
    \caption{Diagrama de clases del sistema}
    \label{fig:clases}
\end{figure}
\clearpage
\subsection{Diagramas de secuencia}

Los diagramas de secuencia reflejan los flujos de interacción entre los distintos componentes del sistema en diferentes escenarios funcionales. Estos esquemas permiten entender cómo se comunican los distintos servicios y agentes en cada proceso clave del sistema.

\paragraph{Gestión de carrito}

En este diagrama se representa el flujo de interacción entre el cliente, la aplicación web y los servicios del catálogo y del carrito.  
El usuario puede añadir, modificar o eliminar productos del carrito, y el sistema actualiza los datos en tiempo real mostrando la información correspondiente a cada producto.

\begin{figure}[H]
    \centering
    \includegraphics[width=0.9\textwidth]{figs/Gestiondecarrito.jpg}
    \caption{Diagrama de secuencia — gestión del carrito}
    \label{fig:carrito}
\end{figure}

\subsubsection{DSGS-01 Gestión de carrito}

\begin{table}[H]
\centering
\small
\begin{tabular}{|p{0.28\textwidth}|p{0.65\textwidth}|}
\hline
\textbf{Nombre} & DSGS-01 Gestión de carrito \\ \hline

\textbf{Actor(es)} & Cliente, WebApp \\ \hline

\textbf{Descripción} & 
Representa el flujo general mediante el cual el cliente gestiona su carrito de compra: 
añadir productos, modificar cantidades o eliminar artículos, mientras la WebApp mantiene 
los datos sincronizados. \\ \hline

\textbf{Precondiciones} & 
El cliente debe disponer de acceso a la WebApp y el catálogo debe estar operativo. \\ \hline

\textbf{Postcondiciones} & 
El carrito queda actualizado correctamente con los productos seleccionados y sus cantidades. \\ \hline

\textbf{Flujo principal} & 
1. El cliente visualiza el catálogo. \newline
2. Selecciona un producto y lo añade al carrito. \newline
3. La WebApp actualiza las unidades y el total del carrito. \newline
4. El cliente modifica cantidades o elimina elementos. \newline
5. El sistema mantiene los datos sincronizados. \\ \hline

\textbf{Flujos alternativos} &
2.a El producto no tiene suficiente stock y se muestra un aviso. \newline
4.a Si el cliente no tiene carrito, el sistema crea uno automáticamente. \\ \hline

\textbf{Requisitos funcionales relacionados} & RF-02, RF-03 \\ \hline
\end{tabular}
\caption{Especificación del flujo DSGS-01 Gestión de carrito.}
\end{table}


\paragraph{Compra estándar}

A continuación, se muestra el flujo completo del proceso de compra.  
Incluye la autenticación del usuario, la creación del checkout, la comunicación con el servicio de pago \textit{Stripe} y la confirmación del pedido.  
También se representa la respuesta automatizada del agente \textit{n8n} para confirmar la compra, descontar el stock y generar la factura.

\begin{figure}[H]
    \centering
    \includegraphics[width=0.95\textwidth]{figs/Compra_estandar_1.png}
    \caption{Diagrama de secuencia — parte 1: autenticación y carrito}
    \label{fig:compra1}
\end{figure}

\subsubsection{DSGS-02 Compra estándar — Parte 1 (autenticación y carrito)}
\begingroup
\setlength{\intextsep}{0pt}%
\setlength{\abovecaptionskip}{0pt}%
\setlength{\belowcaptionskip}{0pt}%
\begin{table}[H]
\centering
\small
\begin{tabular}{|p{0.28\textwidth}|p{0.65\textwidth}|}
\hline
\textbf{Nombre} & DSGS-02 Compra estándar — Parte 1 \\ \hline

\textbf{Actor(es)} & Cliente, WebApp \\ \hline

\textbf{Descripción} & 
Describe el inicio del proceso de compra: autenticación del cliente y revisión del carrito antes de continuar al checkout. \\ \hline

\textbf{Precondiciones} & 
El cliente debe estar registrado y tener acceso a la WebApp. \\ \hline

\textbf{Postcondiciones} & 
El cliente queda autenticado y el carrito listo para proceder al pago. \\ \hline

\textbf{Flujo principal} & 
1. El cliente accede a la WebApp. \newline
2. El sistema solicita autenticación. \newline
3. El cliente introduce sus credenciales. \newline
4. La WebApp valida la autenticación. \newline
5. El cliente revisa el carrito y continúa al checkout. \\ \hline

\textbf{Flujos alternativos} &
4.a Credenciales incorrectas → se muestra un mensaje de error. \\ \hline

\textbf{Requisitos funcionales relacionados} & RF-01, RF-02 \\ \hline
\end{tabular}
\caption{Especificación del flujo DSGS-02 Compra estándar — Parte 1.}
\end{table}
\endgroup



\begin{figure}[H]
    \centering
    \includegraphics[width=0.95\textwidth]{figs/Compra_estandar_2.png}
    \caption{Diagrama de secuencia — parte 2: proceso de checkout y confirmación}
    \label{fig:compra2}
\end{figure}

\subsubsection{DSGS-03 Compra estándar — Parte 2 (checkout y pago)}
\begingroup
\setlength{\intextsep}{0pt}%
\setlength{\abovecaptionskip}{0pt}%
\setlength{\belowcaptionskip}{0pt}%
\begin{table}[H]
\centering
\small
\begin{tabular}{|p{0.28\textwidth}|p{0.65\textwidth}|}
\hline
\textbf{Nombre} & DSGS-03 Compra estándar — Parte 2 \\ \hline

\textbf{Actor(es)} & Cliente, WebApp, Stripe, n8n \\ \hline

\textbf{Descripción} & 
Describe el proceso del checkout: introducción de datos, creación del pedido y procesamiento del pago mediante Stripe. \\ \hline

\textbf{Precondiciones} & 
El cliente debe estar autenticado y disponer de un carrito válido. \\ \hline

\textbf{Postcondiciones} & 
El pago se confirma, el pedido se marca como “pagado” y se notifican las automatizaciones en n8n. \\ \hline

\textbf{Flujo principal} & 
1. El cliente accede al checkout. \newline
2. Introduce datos personales y dirección. \newline
3. Selecciona el método de pago. \newline
4. La WebApp crea el pedido en estado “pendiente”. \newline
5. Stripe procesa el pago. \newline
6. Stripe confirma el pago a la WebApp. \newline
7. La WebApp notifica a n8n para automatizaciones post-compra. \\ \hline

\textbf{Flujos alternativos} &
5.a Stripe requiere autenticación adicional (3D Secure). \newline
7.a Error notificando a n8n → reintento automático. \\ \hline

\textbf{Requisitos funcionales relacionados} & RF-04, RF-05, RF-06 \\ \hline
\end{tabular}
\caption{Especificación del flujo DSGS-03 Compra estándar — Parte 2.}
\end{table}
\endgroup


\begin{figure}[H]
    \centering
    \includegraphics[width=0.95\textwidth]{figs/Compra_estandar_3.png}
    \caption{Diagrama de secuencia — parte 3: flujo alternativo de pago fallido}
    \label{fig:compra3}
\end{figure}

\subsubsection{DSGS-04 Compra estándar — Parte 3 (pago fallido)}
\begingroup
\setlength{\intextsep}{0pt}%
\setlength{\abovecaptionskip}{0pt}%
\setlength{\belowcaptionskip}{0pt}%
\begin{table}[H]
\centering
\small
\begin{tabular}{|p{0.28\textwidth}|p{0.65\textwidth}|}
\hline
\textbf{Nombre} & DSGS-04 Compra estándar — Parte 3 \\ \hline

\textbf{Actor(es)} & Cliente, WebApp, Stripe \\ \hline

\textbf{Descripción} & 
Representa el flujo alternativo donde el pago es rechazado por la pasarela de pago Stripe. \\ \hline

\textbf{Precondiciones} & 
El cliente se encuentra en el proceso de pago. \\ \hline

\textbf{Postcondiciones} & 
El pedido no se completa y se informa al cliente del error. \\ \hline

\textbf{Flujo principal} & 
1. Stripe intenta procesar el pago. \newline
2. Stripe devuelve una respuesta de error. \newline
3. La WebApp muestra un mensaje indicando que el pago ha fallado. \\ \hline

\textbf{Flujos alternativos} &
— No aplica. \\ \hline

\textbf{Requisitos funcionales relacionados} & RF-05 \\ \hline
\end{tabular}
\caption{Especificación del flujo DSGS-04 Compra estándar — Parte 3.}
\end{table}
\endgroup


\paragraph{Generación de factura}

Este diagrama muestra el proceso automatizado de creación de facturas tras la confirmación de un pedido.  
Una vez validado el pago, el sistema genera el documento asociado y lo envía al cliente mediante el agente \textit{n8n}, garantizando así un flujo de trabajo totalmente automatizado.

\begin{figure}[H]
    \centering
    \includegraphics[width=0.9\textwidth]{figs/Generaciondefactura.jpg}
    \caption{Diagrama de secuencia — generación de factura}
    \label{fig:factura}
\end{figure}

\subsubsection{DSGS-05 Generación de factura}
\begingroup
\setlength{\intextsep}{0pt}%
\setlength{\abovecaptionskip}{0pt}%
\setlength{\belowcaptionskip}{0pt}%
\begin{table}[H]
\centering
\small
\begin{tabular}{|p{0.28\textwidth}|p{0.65\textwidth}|}
\hline
\textbf{Nombre} & DSGS-05 Generación de factura \\ \hline

\textbf{Actor(es)} & WebApp, n8n \\ \hline

\textbf{Descripción} & 
Una vez confirmado el pago del pedido, la WebApp notifica a n8n, que genera la factura correspondiente y la envía al cliente. \\ \hline

\textbf{Precondiciones} & 
El pedido debe estar registrado como pagado. \\ \hline

\textbf{Postcondiciones} & 
La factura queda creada, almacenada y enviada al cliente. \\ \hline

\textbf{Flujo principal} & 
1. La WebApp envía un webhook a n8n indicando que el pedido está pagado. \newline
2. n8n genera la factura. \newline
3. n8n almacena o registra la factura en el sistema. \newline
4. n8n envía la factura por correo al cliente. \\ \hline

\textbf{Flujos alternativos} &
2.a Error generando la factura → n8n programa un reintento. \\ \hline

\textbf{Requisitos funcionales relacionados} & RF-07 \\ \hline
\end{tabular}
\caption{Especificación del flujo DSGS-05 Generación de factura.}
\end{table}
\endgroup
\clearpage

\paragraph{Abandono de carrito}

Este diagrama refleja el funcionamiento del proceso de detección y eliminación de carritos inactivos.  
Si un usuario añade productos pero no completa la compra en un tiempo determinado, el sistema marca el carrito como abandonado y libera los productos reservados.

\begin{figure}[H]
    \centering
    \includegraphics[width=0.9\textwidth]{figs/Abandonodecarrito.jpg}
    \caption{Diagrama de secuencia — abandono de carrito}
    \label{fig:abandono}
\end{figure}

\subsubsection{DSGS-06 Abandono de carrito}
\begingroup
\setlength{\intextsep}{0pt}%
\setlength{\abovecaptionskip}{0pt}%
\setlength{\belowcaptionskip}{0pt}%
\begin{table}[H]
\centering
\small
\begin{tabular}{|p{0.28\textwidth}|p{0.65\textwidth}|}
\hline
\textbf{Nombre} & DSGS-06 Abandono de carrito \\ \hline

\textbf{Actor(es)} & WebApp, n8n \\ \hline

\textbf{Descripción} & 
Proceso automatizado que detecta carritos inactivos y libera el stock asociado. \\ \hline

\textbf{Precondiciones} & 
Debe existir un carrito inactivo durante un periodo de tiempo determinado. \\ \hline

\textbf{Postcondiciones} & 
El carrito se marca como abandonado y el stock se libera. \\ \hline

\textbf{Flujo principal} & 
1. n8n ejecuta un flujo programado. \newline
2. Identifica los carritos inactivos. \newline
3. Marca estos carritos como abandonados. \newline
4. Libera stock en la base de datos. \\ \hline

\textbf{Flujos alternativos} &
4.a Error actualizando stock → n8n registra el error. \\ \hline

\textbf{Requisitos funcionales relacionados} & RF-08 \\ \hline
\end{tabular}
\caption{Especificación del flujo DSGS-06 Abandono de carrito.}
\end{table}
\endgroup
\clearpage

\paragraph{Bajo stock}

Por último, se representa el proceso de notificación de stock bajo.  
Cuando el inventario de un producto alcanza un umbral mínimo, el sistema ejecuta una tarea automatizada que envía un aviso al administrador mediante el agente \textit{n8n}, permitiendo reponer el producto a tiempo.

\begin{figure}[H]
    \centering
    \includegraphics[width=0.9\textwidth]{figs/Bajostock.jpg}
    \caption{Diagrama de secuencia — notificación de bajo stock}
    \label{fig:bajo_stock}
\end{figure}

\subsubsection{DSGS-07 Bajo stock}
\begingroup
\setlength{\intextsep}{0pt}%
\setlength{\abovecaptionskip}{0pt}%
\setlength{\belowcaptionskip}{0pt}%
\begin{table}[H]
\centering
\small
\begin{tabular}{|p{0.28\textwidth}|p{0.65\textwidth}|}
\hline
\textbf{Nombre} & DSGS-07 Notificación de bajo stock \\ \hline

\textbf{Actor(es)} & WebApp, n8n, Administrador \\ \hline

\textbf{Descripción} & 
Cuando el nivel de stock de un producto cae por debajo del umbral configurado, n8n envía una notificación al administrador. \\ \hline

\textbf{Precondiciones} & 
Debe existir al menos un producto cuyo stock esté por debajo del mínimo. \\ \hline

\textbf{Postcondiciones} & 
El administrador recibe un aviso para reponer el stock. \\ \hline

\textbf{Flujo principal} & 
1. La WebApp detecta el cambio de stock. \newline
2. Notifica a n8n. \newline
3. n8n verifica si el stock está por debajo del umbral. \newline
4. n8n envía una alerta al administrador. \\ \hline

\textbf{Flujos alternativos} &
— No aplica. \\ \hline

\textbf{Requisitos funcionales relacionados} & RF-09 \\ \hline
\end{tabular}
\caption{Especificación del flujo DSGS-07 Bajo stock.}
\end{table}
\endgroup
\clearpage

\section{Diseño de la base de datos}

El diseño de la base de datos se llevó a cabo inicialmente mediante \textit{dbdiagram.io}, para definir de manera visual las entidades, sus atributos y relaciones.  
Posteriormente, el modelo se validó e implementó en \textit{MySQL Workbench}, comprobando su coherencia con el backend en \textit{Spring Boot}.

\subsection{Modelo entidad–relación}

El modelo E/R del sistema representa las entidades principales (\textit{usuarios, productos, carritos, pedidos, pagos, facturas}) y las relaciones entre ellas.  
Estas relaciones garantizan la integridad referencial mediante el uso de claves primarias, foráneas y restricciones de unicidad.
\clearpage  

\subsection{Modelo relacional y prototipo inicial}

El modelo relacional resultante se compone de tablas como \texttt{usuarios}, \texttt{productos}, \texttt{carritos}, \texttt{carrito\_items}, \texttt{pedidos}, \texttt{pagos} y \texttt{facturas}.  
La Figura~\ref{fig:prototipoBD} muestra el esquema final diseñado.

\begin{figure}[H]
    \centering
    \includegraphics[width=0.95\textwidth]{figs/prototipoBD.png}
    \caption{Prototipo de la base de datos relacional}
    \label{fig:prototipoBD}
\end{figure} 
\clearpage
\section{Diseño de la interfaz de usuario}

Además del diseño lógico y técnico del sistema, se ha desarrollado un prototipo visual que representa la estructura y apariencia de la aplicación web. 
Este diseño tiene como objetivo definir la disposición de los elementos de la interfaz, mejorar la experiencia de usuario y servir como guía en la fase de implementación del \textit{frontend}.

\subsection{Prototipo de la página de inicio}

La Figura~\ref{fig:prototipo_inicio} muestra el prototipo correspondiente a la página principal de la aplicación, diseñado con la herramienta \textit{Pencil Project}. 
La interfaz se ha elaborado siguiendo criterios de claridad, jerarquía visual y coherencia con el resto del sistema.

En la parte superior se encuentra la barra de navegación, que incluye el logotipo de la tienda, el buscador, las secciones principales y el acceso al carrito. 
A continuación, se muestra un banner promocional con un mensaje destacado y un botón de acción. 
Debajo se encuentran las categorías principales de productos, seguidas de una sección con artículos destacados y, finalmente, un pie de página con la información corporativa y enlaces a redes sociales.

\begin{figure}[H]
    \centering
    \includegraphics[width=\textwidth, keepaspectratio]{figs/prototipoinicio.png}
    \caption{Prototipo de la página de inicio de la aplicación web.}
    \label{fig:prototipo_inicio}
\end{figure}
\clearpage
\subsection{Prototipo del catálogo de productos}

La Figura~\ref{fig:prototipo_catalogo} muestra el prototipo de la vista de catálogo, en la que el usuario puede explorar el inventario completo. 
La interfaz mantiene la coherencia visual con la página de inicio (mismo encabezado y pie de página) e incorpora una barra de filtros en la parte superior con búsqueda por texto, selección de categoría, rango de precio, opción de disponibilidad (\textit{En stock}) y un control de \textit{Ordenar por}. 
Bajo dicha barra se presenta una rejilla de productos (tarjetas con imagen, nombre, precio y botón \textit{Añadir al carrito}) organizada en filas y columnas para facilitar el escaneo visual.

\begin{figure}[H]
    \centering
    \includegraphics[width=\textwidth, keepaspectratio]{figs/catalogo_de_productos.png}
    \caption{Prototipo del catálogo de productos con barra de filtros y rejilla de artículos.}
    \label{fig:prototipo_catalogo}
\end{figure}
\clearpage
\subsection{Prototipo del carrito de compra}

La Figura~\ref{fig:prototipo_carrito} muestra el prototipo de la vista de carrito. 
La interfaz mantiene la coherencia visual con el resto de pantallas (mismo encabezado y pie) y presenta una \textit{tabla de líneas} donde se visualiza cada producto añadido con su imagen, nombre, precio unitario, selector de cantidad y subtotal, además de la acción de eliminar. 
En la parte inferior se destaca el importe \textbf{Total} y se incluyen los botones de navegación principales: \textit{Seguir comprando} y \textit{Finalizar compra}.

\begin{figure}[H]
    \centering
    \includegraphics[width=\textwidth, keepaspectratio]{\string figs/carrito_de_compra.png}
    \caption{Prototipo del carrito de compra con líneas de producto, total y acciones principales.}
    \label{fig:prototipo_carrito}
\end{figure}
\clearpage

\subsection{Prototipo del proceso de pago (\textit{Checkout})}

La Figura~\ref{fig:prototipo_checkout} muestra el prototipo correspondiente a la vista de pago del sistema. 
Esta pantalla representa la fase final del proceso de compra y mantiene la coherencia visual con el resto de interfaces, reutilizando la barra de navegación superior y el pie de página.

La interfaz se estructura en distintos bloques claramente diferenciados: \textbf{Datos personales}, \textbf{Dirección de envío}, \textbf{Método de pago} y \textbf{Tarjeta}. 
Cada sección incluye los campos necesarios para introducir la información correspondiente, priorizando la claridad y la usabilidad. 
En la parte inferior derecha se presenta el \textbf{resumen del pedido}, donde se muestran los productos seleccionados, el desglose de precios (subtotal, envío y total) y las acciones finales: 
\textit{Volver al carrito} y \textit{Confirmar pedido}.

\begin{figure}[H]
    \centering
    \includegraphics[width=\textwidth, keepaspectratio]{\string figs/checkout_pago.png}
    \caption{Prototipo del proceso de pago con formularios y resumen del pedido.}
    \label{fig:prototipo_checkout}
\end{figure}

\chapter{Implementación y pruebas}
\input{tex/implementacion_pruebas.tex}
\chapter{Conclusiones}
\input{tex/conclusiones.tex}



\pagestyle{appendix}

\appendix
\chapter{Apéndice}
\input{tex/apendice.tex}

\addcontentsline{toc}{chapter}{Bibliografía}
\bibliographystyle{unsrt}
\bibliography{bib/bibliografia}




\end{document}

%%% Local Variables:
%%% mode: latex
%%% TeX-master: t
%%% End:
